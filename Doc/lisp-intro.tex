\documentclass[a4paper]{amsart}

\usepackage{fancyvrb}
\newcommand{\prints}{\Rightarrow}
\newcommand{\eval}{\rightsquigarrow}
\newcommand{\lsq}{[} 
\newcommand{\rsq}{]}

\newcommand{\Poem}{\textsc{Poem}}

\DefineVerbatimEnvironment{Code}{Verbatim}{%
  gobble=2,%
  commandchars=\\\[\],%
  codes={\catcode`$=3\catcode`^=7\catcode`_=8}
}

%%% Local Variables:
%%% mode: LaTeX
%%% TeX-master: "lisp-intro"
%%% End:


\begin{document}
\title{A Whirlwind Introduction to ALisp}
\author{Matthias H\"olzl}

\maketitle

\section{Lisp}
\label{sec:lisp}


\subsection{Values}
\label{sec:values}

Lisp contains most of the usual kinds of values: numbers, strings,
characters, etc.  Some literals are written differently than in other
languages.  Fig.~\ref{fig:value-literals} shows some examples.

The boolean values are written \texttt{nil} and \texttt{t}; in
conditionals, values of an arbitrary type can appear, and every value
different from \texttt{nil} counts as a true value.  Proper lists are
either the empty list or a pair consisting of a first value (called
the \texttt{car} of the list) and a second value which is itself a
list (called the \texttt{cdr} and pronounced ``could-er'').  A
peculiarity of Common Lisp is that the empty list \texttt{()} is
identical to the false value \texttt{nil} (which is also a symbol).
Furthermore, the empty list always evaluates to itself.  Therefore
\texttt{()} and \texttt{nil} can be used interchangeably, but it is
good style to write \texttt{()} when the value is used as list and
\texttt{nil} to denote the boolean value.

Common Lisp uses packages to partition the namespace.  Packages can
export symbols, thereby making them visible to other packages.  There
is always a notion of the ``current package'' which is the package
that is used to intern symbols that do not include a package prefix.

Symbols are similar to ``interned strings'' in Java or ``unique
strings'' in other programming languages, except that symbols are
interned in packages. Like strings, symbols consist of a sequence of
characters; unlike strings, symbols are immutable and two symbols in
the same package are identical whenever their lexical representation
is the same.  To access an exported symbol in a package that is not
directly visible from the current package it can be prefixed with the
package name, i.e., to access the symbol \texttt{bar} in package
\texttt{foo}, write \texttt{foo:bar}.  If a symbol is not exported, it
can still be accessed by separating the symbol and package names with
two colons: \texttt{foo::bar} accesses symbol \texttt{bar} in package
\texttt{foo} even when it is not exported.  (This should only be used
for debugging purposes).

Typically, symbols are written without any delimiters; they can
contain a wide range of characters: letters, number, \verb|-|,
\verb|+|, \verb|*|, etc. However, if a symbol contains spaces (or
other whitespace) it must be enclosed in vertical bars (\verb=|=)
otherwise it would be interpreted as several consecutive symbols.
Unless a symbol is enclosed in vertical bars, the Lisp reader converts
all letters in the symbol to upper case; therefore Lisp behaves like a
case insensitive language (even though it is in reality case
sensitive).

Symbols and lists play an important role in Common Lisp: In Lisp,
programs are themselves represented as Lisp data, with symbols and
lists serving as representations for variables and function calls,
respectively.

\begin{figure}[tp]
  \centering
  \begin{tabular}{|l|l|}
    \hline
    Type& Example Literals\\
    \hline\hline
    Boolean & \texttt{nil}, \texttt{t}\\
    \hline
    Number & \texttt{1}, \texttt{-123}, \texttt{1.23e2}\\
    \hline
    String & \verb|"This is a string"|\\
    \hline
    Symbol & \verb|print|, \verb|a-1|, \verb|*var*|, \verb=|Symbol with space|=\\
    \hline
    Symbol in Package \texttt{a} & \verb|a:my-symbol|, \verb|a::internal|\\
    \hline
    Keyword & \verb|:my-keyword|\\
    \hline
    Character & \verb|#\A|, \verb|#\Newline|\\
    \hline
    List & \verb|()|, \verb|(1 2 3)|, \verb|(print "Hello")|\\
    \hline
    Vector/Array & \verb|#(1 2 3)|, \verb|#((1 2) (3 4))|\\
    \hline
  \end{tabular}
  \caption{Values}
  \label{fig:value-literals}
\end{figure}

\subsection{Evaluation}
\label{sec:evaluation}

The source code of a Lisp program is stored in files, typically with
the suffix \verb|.lisp| or \verb|.cl|.  When the Lisp system processes
a file it first converts the textual representation into a Lisp data
structure; this data structure is then evaluated according to the
following rules:\footnote{These rules are not really correct, but they
  should be precise enough to understand most programs.}
\begin{itemize}
\item All objects except lists and symbols evaluate to themselves.
\item Symbols represent (global or local) variables and are looked up
  in the variable environment.
\item Lists represent function calls, macros or applications of
  built-in operators (so-called \emph{special forms}).  Function calls
  are evaluated in the following manner:
  \begin{itemize}
  \item Each element of the list is recursively evaluated.  If the
    first element of the list is a symbol, its value is looked up in
    the \emph{function environment}, all other symbols directly
    appearing in the list are looked up in the \emph{variable
      environment}.
  \item When all elements of the list are evaluated, the value of the
    first argument (a function) is applied to the other arguments.
  \item The evaluation of a function call results in zero, one or more
    values.
  \end{itemize}
\item Each special form has its own evaluation rules.  For example,
  \texttt{lambda}-forms evaluate to functions.
\end{itemize}

To prevent a symbol or list from being evaluated it can be prefixed by
an apostrophe (\verb|'|).  For example \verb|'x| evaluates to the
symbol \texttt{x}, not to the value of the variable \texttt{x}.
Similarly \verb|'(print "x")| evaluates to a list with two elements
(the symbol \texttt{print} and the string \verb|"x"|), not to a
function call.  As a rule, lists that are used as data structures must
always be preceeded by an apostrophe: \verb|'(1 2 3)| evaluates to a
list consisting of three integers, \verb|(1 2 3)| leads to an error
message, since Lisp tries to evaluate this form as a function call and
\texttt{1} is not a valid function name.

\subsection{Global Variables and Functions}
\label{sec:glob-vari-funct}

Global variables are defined with the operators \texttt{defvar} and
\texttt{defparameter}.  By convention, the names of global variables
start and end with an asterisk (\verb|*|).  For example:
\begin{Code}
  (defvar *my-var* 123)
  (defparameter *my-other-var* 234)
  *my-var*                         $\eval$ 123
  *my-other-var*                   $\eval$ 234
\end{Code}
The value of global variables can be changed with the \texttt{setf}
operator:
\begin{Code}
  (setf *my-var 345)
  (setf *my-other-var* 456)
  *my-var*                         $\eval$ 345
  *my-other-var*                   $\eval$ 456
\end{Code}
The difference between \texttt{defvar} and \texttt{defparameter} is
that a \texttt{defvar}-form does not overwrite an existing value for
the variable whereas \texttt{defparameter} does.  Thus, if we continue
the example:
\begin{Code}
  (defvar *my-var* 123)
  (defparameter *my-other-var* 234)
  *my-var*                         $\eval$ 345
  *my-other-var*                   $\eval$ 234  
\end{Code}

Local variables can be bound with \texttt{let} and \texttt{let*}
forms.  The difference between these forms is that \texttt{let} binds
all variables in parallel (so that none of the freshly introduced
bindings is visible on the right hand sides) whereas \texttt{let*}
binds the variables sequentially:
\begin{Code}
  (let ((x 1)
        (y 2))
    (list x y))                                   $\eval$ (1 2)

  (let* ((x 1)
         (y (+ x 1)) ; not possible with let
    (list x y))                                   $\eval$ (1 2)
\end{Code}
This code also shows a comment; comments start after a semocolon
(\verb|;|) and end at the end of the line.

Functions are defined using the \texttt{defun} form:
\begin{Code}
  (defun my-fun (x)
    (print x))
\end{Code}
This expression defines a function called \texttt{my-fun} (i.e., it
binds the function variable \texttt{my-fun} to the corresponding
function object), that takes one argument.  When this function is
called it calls the \texttt{print} function to print the value of its
argument:
\begin{Code}
  (my-fun 123)
  $\prints$ 123
  $\eval$ 123
\end{Code}
Each function returns the last value in its body; since the
\texttt{print} function returns its argument after printing it, a call
to \texttt{my-fun} also returns its argument.  The \texttt{values}
special operator can be used to return zero, one or more values:
\begin{Code}
  (defun zero-values (x y)
    (print x)
    (print y)
    (values))

  (defun three-values (x y)
    (print x)
    (print y)
    (values x y x))
\end{Code}
Calling \texttt{zero-values} with arguments \texttt{1} and \texttt{2},
i.e., \texttt{(zero-values 1 2)}, prints \texttt{1} and \texttt{2} and
returns no values, a call \texttt{(three-values 1 2)} prints
\texttt{1} and \texttt{2} and returns the three values \texttt{1},
\texttt{2} and \texttt{1}.  Multiple values can be bound with the form
\texttt{multiple-value-bind}:
\begin{Code}
  (multiple-value-bind (a b c) (three-values 1 2)
    (print a)
    (print b)
    (print c))
\end{Code}
This code will print 5 lines of output: \texttt{1}, \texttt{2},
\texttt{1}, \texttt{2} and \texttt{1}.  The first two lines are from
the call to \texttt{three-values}, the last three lines from the
\texttt{print} statements in the body of \texttt{multiple-value-bind}.

In addition to the required arguments, functions can take
\texttt{optional}, \texttt{keyword} and \texttt{rest} arguments.
Optional arguments need not be provided by the caller.  The function
definition can specify a default value for optional elements that are
not provided, if no default value is specified, \texttt{nil} is used:
\begin{Code}
  (defun opt-arg (x &optional (y 1) z)
    (format t "~&~A, ~A, ~A" x y z))
  (opt-arg 'a)             $\prints$ A, 1, NIL
  (opt-arg 'a 'b)          $\prints$ A, B, NIL
  (opt-arg 'a 'b 'c)       $\prints$ A, B, C
\end{Code}
The function \texttt{format} prints formatted output to a stream.
Here the stream is specified as \texttt{t}, denoting the standard
 output.  The format directive \verb|~&| is a (conditional) newline,
\verb|~A| takes the next unprocessed argument and prints it.  In the
first call to \texttt{opt-arg}, the call provides no values for the
variables \texttt{y} and \texttt{z}, so their default values are
used.  In the second call a value is provided for \texttt{y} but not
,for \texttt{z}, in the last call, all variables are provided values
by the caller.

If a function takes many arguments, calls to the function can be
rather confusing.  Keyword arguments can be used to clarify the roles
of the arguments: like optional arguments keyword arguments need not
be provided by the caller, but in calls, keyword arguments are
explicitly named by a keyword and can therefore be provided in any
order:
\begin{Code}
  (defun key-arg (x &key (y 1) z)
    (format t "~&~A, ~A, ~A" x y z))
  (key-arg 'a)                    $\prints$ A, 1, NIL
  (key-arg 'a :y 'b)              $\prints$ A, B, NIL 
  (key-arg 'a :y 'b :z 'c)        $\prints$ A, B, C 
  (key-arg 'a :z 'c :y 'b)        $\prints$ A, B, C 
\end{Code}

\subsection{Structs and Classes}
\label{sec:structs-classes}

Common Lisp contains CLOS (Common Lisp Object System), an extremely
powerful object system.  There are two different class-like data
types: Structures (also called structure-classes) and classes.
Structures as well as classes contain only data members; methods are
defined outside of the user-defined data types.

Structures have several restrictions that make them much less flexible
than classes.  For example, classes can be redefined (and existing
instances using the old class definition can be upgraded to the new
class definition), classes support multiple inheritance, and the class
of a class-instance can be changed dynamically.  Structures provide
none of these features, but they can therefore be implemented more
efficiently than classes.  Therefore it is advisable to use structure
types only in situation where performance considerations are
paramount.

Unfortunately, for historic reasons, the syntactic for of class and
structure definitions is quite different.  Structures are defined in
the following way:
\begin{Code}
  (defstruct (simple-state (:conc-name #:simple-))
    (start-loc '(0 0) :type list)
    robot-loc
    env)  
\end{Code}
This defines a structure-class \texttt{simple-state} with three
instance variables that can be accessed usin functions called
\texttt{simple-start-loc}, \texttt{simple-robot-loc} and
\texttt{simple-env}.  The prefix of the instance accessors is defined
by the \texttt{:conc-name} struct-option.  Instances are created using
the constructor \texttt{make-simple-state} which takes a keyword
argument for each instance variable.  The instance variable
\texttt{start-loc} is additionally provided with a default value and
restricted to values of type \texttt{list}.  The call
\begin{Code}
  (make-simple-state :start-loc '(1 2) :end-loc '(5 7) :env *my-env*)
\end{Code}
creates a new instance of type \texttt{simple-state} in which the
instance variables (which are typically called \emph{slots} in Lisp)
are initialized to the given values.

A class definition has the following form:
\begin{Code}
  (defclass <simple-env> (<fully-observable-env> <grid-world>)
    ((move-success-prob :type float
                        :initarg :move-success-prob :initform 0.95
                        :accessor move-success-prob)
     (wall-collision-cost :initarg :wall-collision-cost :initform 0.5
                          :accessor wall-collision-cost)))
\end{Code}
This defines a class \verb|<simple-env>| (the use of angle brackets
around the class name is a convention that is used in ALisp and has no
further significance).  This class inherits from two superclasses,
\verb|<fully-observable-env>| and \verb|<grid-world>| and has two
slots \texttt{move-success-prob} and \texttt{wall-collision-cost}.
The name of the accessor functions and the keyword arguments for the
constructor have to be explicitly specified using the
\texttt{:accessor} and \texttt{:initarg} keyword arguments in the slot
specification; furthermore each slot specifies a default value
(\texttt{:initform}); the \texttt{move-success-prob} slot additionally
specifies the type of values it can store.  The accessors of classes
defined with \texttt{defclass} are used as specified, no prefix is
appended.

Instances of classes are created using the \texttt{make-instance}
function, which takes the keywords specified in the class definition
and the keywords inherited from superclasses.  For example:
\begin{Code}
  (make-instance '<simple-env> :move-success-prob 0.8)
\end{Code}
(Note that we pass the \emph{name} of the class as first argument to
\texttt{make-instance}, i.e., the first argument to
\texttt{make-instance} is typically quoted.)

\subsection{Generic Functions and Methods}
\label{sec:gener-funct-meth}

Methods do not belong to classes, instead they are grouped into
\emph{generic functions}.  A generic function can explicitly be
defined with \texttt{defgeneric} or implicitly by a method definition.

The code
\begin{Code}
  (defgeneric description (thing))
\end{Code}
defines a generic function \texttt{description} that takes a single
argument.  Methods can be specialized on this argument:
\begin{Code}
  (defmethod description (thing)
    (print "Some unspecified thing"))
  (defmethod description ((l list))
    (print "A list"))
  (defmethod description ((n number))
    (print "A number"))
  (defmethod description ((st simple-state))
    (print "Our very own state"))
  (defmethod description ((env <simple-env>))
    (print "A simple environment"))

  (description "Foo")                        $\prints$ Some unspecified thing
  (description '())                          $\prints$ A list
  (description 123)                          $\prints$ A number
  (description (make-simple-state))          $\prints$ Our very own state
  (description (make-instance <simple-env>)) $\prints$ A simple environment
\end{Code}
Note that methods can be defined on ``primitive'' types (such as
number), on structures and on classes.  It is even possible to define
methods specialized on single objects:
\begin{Code}
  (defmethod description ((n (eql 0)))
    (print "Naught"))
  (description 0)                            $\prints$ Naught
\end{Code}

Generic functions support multi-dispatch, i.e., they can be specified
on several arguments:
\begin{Code}
  (defmethod multi (x y)
    (list x y))
  (defmethod multi ((x number) (y number))
    (+ x y))
  (defmethod multi ((x string) (y string))
    (concatenate 'string x y))
  (multi  'a  'b)                               $\eval$ (A B)
  (multi 'a   1)                                $\eval$ (A 1)
  (multi  1   2)                                $\eval$ 3
  (multi 'a  "b")                               $\eval$ (A "b")
  (multi "a" 'b)                                $\eval$ ("a" B)
  (multi "a" "b")                               $\eval$ "ab"
\end{Code}

\section{The Simple-Waste Example}
\label{sec:simple-waste}

The Simple-Waste example is meant to introduce ALisp using the simple
scenario of a robot moving toward a target in an arena in which there
may be some obstacles.  The following sections assume that your Lisp
environment is already correctly set up.

\subsection{Running the Simple-Waste Example}
\label{sec:running-simple-waste}

In the Lisp listener (REPL), evaluate the following forms:
\begin{Code}
  CL-USER> (asdf:load-system :waste)
  \lsq...\rsq
  CL-USER> (in-package :simple-prog)
  #<PACKAGE "SIMPLE-PROG">
  SIMPLE-PROG> (explore-environment)
  Welcome to the simple robot example.

  This environment demonstrates a robot that moves around on a
  rectangular grid, until it reaches a target area.  X's on the map
  represent walls, blank spaces are roads.  The robot is represented
  by 'r'.  You can move by entering N, E, S, W.  To quit the
  environment, enter NIL.  (All input can be in lower or upper case.)

  Last observation was 
  XX0X1X2X3XXX
  0X        0X
  1X  rr    1X
  2X        2X
  XX0X1X2X3XXX
  Target: (0 0)
  Action?
\end{Code}
The \texttt{load-system} form loads the \texttt{:waste} system
definition which contains, among others, the code for the Simple-Waste
example.  The \texttt{explore-environment} runs a function that allows
you to interactively explore the environment.  The start position of
the robot is randomly generated and may be different for each
execution.  After completing an episode or entering \texttt{nil} at
the prompt, you can start the reinforcement learner by calling
\texttt{learn-behavior}:
\begin{Code}
  SIMPLE-PROG> (learn-behavior)
  Learning behavior using random exploration strategy
  Learning
  Episode 0.
  NIL
\end{Code}
The \texttt{learn-behavior} function calls the primitive
\texttt{learn} function with parameters that are controlled by its
keyword arguments and some global variables.  After learning is
completed, you can evaluate the performance of the learned policy
against environments with randomly genereated robot start positions:
\begin{Code}
  SIMPLE-PROG> (evaluate-performance)

  Learning curves for HORDQ-A-1, HORDQ-A-2 are:
  Evaluating policies..................................................
  Evaluating policies..................................................
  #((#(-2.36 4.71 4.71 4.72 4.73 4.7 4.7
       4.73 4.72 4.72 4.71 4.73 4.73 4.71
       4.71 4.72 4.69 4.72 4.71 4.69 4.7
       4.71 4.72 4.71 4.68 4.71 4.7 4.7
       4.72 4.72 4.71 4.73 4.72 4.72 4.73
       4.69 4.71 -9.66 -8.93 -8.96 4.72
       4.71 4.72 4.71 4.74 4.7 4.72 4.71 4.71 4.71))
    (#(-3.52 1.31 3.66 4.53 2.06 0.86 1.27
       2.62 4.74 4.72 4.72 4.71 4.69
       4.72 4.7 4.69 4.65 4.72 4.67 4.68
       4.67 4.68 4.68 4.69 4.72 4.7 4.71
       4.7 4.72 4.73 4.73 4.72 4.72 4.69
       4.71 4.72 4.71 4.71 4.69 4.72 4.73
       4.72 4.72 4.71 4.69 4.69 4.72 4.72 4.71 4.71)))
  ; No value
\end{Code}
The function \texttt{learn-behavior} stores copies of the policies
learned after having completed $2\%$, $4\%$, \dots, $100\%$ of the
steps in the training run.  \texttt{evaluate-performance} runs each of
these policies against randomly generated examples and returns a
vector of the resulting scores.  In the example, the reward for
reaching the target field is $5.0$ and the cost for each step is
$0.1$.  The cost for bumping into a wall is $0.5$.  To complicate the
problem, the robot only moves into the desired direction with $90\%$
probability, and perpendicular to this direction otherwise.  Hence,
the best expected score in the environment is $\approx4.7$, which
both algorithms achieve frequently.  Not that the first algorithm
rapidly achieves this score for most of the runs, but several runs
toward the end of the learning curve show severely degraded
performance.  The second algorithm converges less rapidly, but
consistently stays near the maximum performance after about a quarter
of all tries.  We will see why the algorithms exhibit this behavior
when we look at their implementation.

Let's try a slightly more involved example:
\begin{Code}
  SIMPLE-PROG> (learn-behavior :environment-type :medium
                               :use-complex-environment t)
  Learning behavior using random exploration strategy
  Learning
  Episode 0........................................
  NIL

  SIMPLE-PROG> (evaluate-performance)
  Learning curves for HORDQ-A-1, HORDQ-A-2 are:
  Evaluating policies..................................................
  Evaluating policies..................................................
  #((#(-6.85 -21.16 -3.87 -3.01 -3.48 -4.41
       -3.73 -21.15 -4.03 -21.56 -4.09
       -4.51 -3.82 -23.64 -24.11 -4.12 -3.19
       -4.31 -3.65 -3.09 -3.83 -2.48
       -2.56 -3.94 -2.68 -3.86 -2.29 -21.24
       -4.26 -4.21 -4.51 -4.51 -3.49
       -21.61 -3.23 -2.07 -2.98 -2.04 -2.33
       -2.46 -2.03 -2.93 -2.41 -4.12
       -4.35 -3.94 -4.61 -2.89 -4.6 -4.12))
    (#(-6.56 -6.23 -4.18 -4.44 -4.35 -4.28
       -4.07 -4.02 -4.12 -4.32 -4.02
       -3.73 -4.04 -0.16 -0.54 0.04 -0.47
       0.42 -0.37 1.08 1.1 1.39 0.26 0.7
       0.35 -0.66 0.49 1.07 2.91 3.95 2.96
       2.61 2.5 3.67 3.1 3.05 3.36 3.57
       3.49 3.99 2.84 3.82 3.69 3.9 3.58
       3.97 3.98 4.08 3.97 4.02)))
  ; No value

  SIMPLE-PROG> (explore-environment)
  Welcome to the simple robot example.
  
  This environment demonstrates a robot that moves around on a
  rectangular grid, until it reaches a target area.  X's on the map
  represent walls, blank spaces are roads.  The robot is represented
  by 'r'.  You can move by entering N, E, S, W.  To quit the
  environment, enter NIL.  (All input can be in lower or upper case.)
  Last observation was 
  XX0X1X2X3X4X5X6X7XXX
  0X      XX        0X
  1X      XX        1X
  2XXXXX  XX        2X
  3X      XX        3X
  4X        rr      4X
  5X                5X
  6X                6X
  7X                7X
  XX0X1X2X3X4X5X6X7XXX
  Target: (0 0)
  Action? nil
\end{Code}
Here we explore a medium-sized environment with a slightly more
complex structure, with the same reward structure as before.  To
compensate for this increased complexity, \texttt{learn-behavior} runs
the experiment with a larger number of steps, as can be seen from its
output (a dot is printed for every $2500$ steps).  The evaluation
shows that in this case the first algorithm does not learn a useful
behavior, whereas the second algorithm again converges to a nearly
optimal policy (taking into account the increased size of the arena
and the additional movement steps necessary to drive around the wall
when starting in the upper right quadrant).  To investigate the
behaviors of policies it is sometimes useful to observe them in
action.  This can be done by calling \texttt{(explore-policies)}:
\begin{Code}
  SIMPLE-PROG> (explore-policies)
  Welcome to the simple robot example.
  \lsq...\rsq
  Env state: 
  XX0X1X2X3X4X5X6X7XXX
  0X      XX        0X
  1X      XX        1X
  2XXXXX  XXrr      2X
  3X      XX        3X
  4X                4X
  5X                5X
  6X                6X
  7X                7X
  XX0X1X2X3X4X5X6X7XXX
  Target: (0 0)
  Stack: ((NAV NAVIGATE-CHOICE ((LOC 0 0))) (TOP NAV NIL))>
  Set of available choices is (N E S W)
  --------------------------------------------------
  Advisor 0: 
  Componentwise Q-values are
    #((N (Q -74.13) (QR -0.2) (QC -73.93) (QE 0.0))
      (E (Q -74.1) (QR -0.17) (QC -73.93) (QE 0.0))
      (S (Q -74.03) (QR -0.11) (QC -73.93)(QE 0.0))
      (W (Q -74.03) (QR -0.1) (QC -73.93) (QE 0.0)))
  Recommended choice is W
  --------------------------------------------------
  Advisor 1: 
  Componentwise Q-values are
    #((N (Q -0.79) (QR -0.1) (QC -0.69) (QE 0.0))
      (E (Q -0.79) (QR -0.1) (QC -0.69) (QE 0.0))
      (S (Q -0.61) (QR -0.1) (QC -0.51) (QE 0.0))
      (W (Q -0.71) (QR -0.1) (QC -0.61) (QE 0.0)))
  Recommended choice is S
  --------------------------------------------------
  Please enter choice, or nil to terminate. 
\end{Code}
Like \texttt{explore-environment}, the function
\texttt{explore-policies} allows us to interact with the environments.
In addition it shows the $Q$-functions computed by the different
learning algorithms and their choice.  After following the advice of
the first algorithm for a few steps we see why it performs badly on
many examples: even though the robot bumps into the wall when moving
west, the algorithm repeatedly suggests this move.  The second
algorithm, in contrast, correctly suggests moving south.  (When
looking at the code we will see that the first algorithm operates
without knowing the robot's position on the board or the walls
adjacent to the robot's position so that it cannot develop a strategy
that behaves sensibly if there are obstacles in the arena.)  As a
final example, let us try the second algorithm on a maze-like example:
\begin{Code}
  SIMPLE-PROG> (learn-behavior :environment-type :maze :algorithm-names '(hordq-a-2))
  Learning behavior using random exploration strategy
  Learning
  Episode
  0...........................................................
   ...........................................................
   ...........................................................
   ...........................................................
   ...........................................................
   ...........................................................
   ..............................................
  NIL

  SIMPLE-PROG> (evaluate-performance)  
  Learning curves for HORDQ-A-2 are:
  Evaluating policies..................................................
  #((#(-7.63 -7.08 -4.17 -4.15 -2.31 -1.45
       -1.17 -0.27 0.75 0.82 1.61 2.89
       2.26 2.79 2.45 3.01 2.78 2.78 2.96
       2.94 2.78 2.8 2.9 2.89 2.77 2.66
       3.01 2.92 2.68 2.96 2.82 2.83 2.87
       2.87 2.92 2.64 2.77 2.87 2.59 2.75
       2.95 2.82 3.13 2.7 3.07 2.65 2.87 3.02 2.73 2.77)))
  ; No value

  SIMPLE-PROG> (explore-policies)
  \lsq...\rsq
  XX0X1X2X3X4X5X6X7X8X9XXX
  0X      XX            0X
  1X      XXXXXXXXXX  XX1X
  2XXXXX  XX  XX        2X
  3X      XX  XXXXXXXX  3X
  4X  XXXXXX  XX        4X
  5X          XX  XXXXXX5X
  6XXXXXXXXX  XX    rr  6X
  7X  XX      XXXX      7X
  8X  XX  XX  XX  XX    8X
  9X      XX            9X
  XX0X1X2X3X4X5X6X7X8X9XXX
  Target: (0 0)
\end{Code}
Since the average path length in this environment is $\approx 14$
steps, the policy is again close to the optimal one.

\subsection{Diving into the Code}
\label{sec:diving-into-code}

Let us now look at the code of the Simple-Waste example.  The system
definition is in the file \texttt{Sources/waste.asd}.

\end{document}

